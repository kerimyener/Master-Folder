\chapter*{Abstract}
\addcontentsline{toc}{chapter}{Abstract}
One difficulty involved with autonomous driving is ensuring that the cars have proper navigation ability in cases of unreliable localization. Hence, robot localization is one of the fundamental problems that must be resolved to achieve full driving autonomy, bearing in mind that human drivers generally have prior knowledge of an area to be traveled and are prepared to safely navigate a car. In the same way, self-driving cars need to know where they are in order to determine what to do next. Therefore, to guarantee the achievement of autonomy, vehicles must be able to interact with their environment by receiving information via sensors. For this reason, self-driving vehicles are, in most cases, equipped with several sensors, such as light detection and ranging (LIDAR), an inertial measurement unit (IMU), and a global positioning system (GPS), among others.
\par With these factors in mind, this thesis focuses on the different localization methods that are available for self-driving cars. These range from basic to advanced and include dead reckoning, sensor fusion, by means of an extended Kalman filter (EKF), and scan matching methods, including normal distributions transform (NDT) and iterative closest point (ICP). These methods are also typically combined, so that each method can help compensate for the limitations of the other methods. Moreover, these methods have been applied to the MIA car, which was used by the German Research Center for Artificial Intelligence (DFKI) to evaluate the results of autonomous tests in order to suggest which localization methods may be most suitable for helping a self-driving vehicle localize its surroundings.





